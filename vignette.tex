\documentclass{article}\usepackage[]{graphicx}\usepackage[]{color}
%% maxwidth is the original width if it is less than linewidth
%% otherwise use linewidth (to make sure the graphics do not exceed the margin)
\makeatletter
\def\maxwidth{ %
  \ifdim\Gin@nat@width>\linewidth
    \linewidth
  \else
    \Gin@nat@width
  \fi
}
\makeatother

\definecolor{fgcolor}{rgb}{0.345, 0.345, 0.345}
\newcommand{\hlnum}[1]{\textcolor[rgb]{0.686,0.059,0.569}{#1}}%
\newcommand{\hlstr}[1]{\textcolor[rgb]{0.192,0.494,0.8}{#1}}%
\newcommand{\hlcom}[1]{\textcolor[rgb]{0.678,0.584,0.686}{\textit{#1}}}%
\newcommand{\hlopt}[1]{\textcolor[rgb]{0,0,0}{#1}}%
\newcommand{\hlstd}[1]{\textcolor[rgb]{0.345,0.345,0.345}{#1}}%
\newcommand{\hlkwa}[1]{\textcolor[rgb]{0.161,0.373,0.58}{\textbf{#1}}}%
\newcommand{\hlkwb}[1]{\textcolor[rgb]{0.69,0.353,0.396}{#1}}%
\newcommand{\hlkwc}[1]{\textcolor[rgb]{0.333,0.667,0.333}{#1}}%
\newcommand{\hlkwd}[1]{\textcolor[rgb]{0.737,0.353,0.396}{\textbf{#1}}}%
\let\hlipl\hlkwb

\usepackage{framed}
\makeatletter
\newenvironment{kframe}{%
 \def\at@end@of@kframe{}%
 \ifinner\ifhmode%
  \def\at@end@of@kframe{\end{minipage}}%
  \begin{minipage}{\columnwidth}%
 \fi\fi%
 \def\FrameCommand##1{\hskip\@totalleftmargin \hskip-\fboxsep
 \colorbox{shadecolor}{##1}\hskip-\fboxsep
     % There is no \\@totalrightmargin, so:
     \hskip-\linewidth \hskip-\@totalleftmargin \hskip\columnwidth}%
 \MakeFramed {\advance\hsize-\width
   \@totalleftmargin\z@ \linewidth\hsize
   \@setminipage}}%
 {\par\unskip\endMakeFramed%
 \at@end@of@kframe}
\makeatother

\definecolor{shadecolor}{rgb}{.97, .97, .97}
\definecolor{messagecolor}{rgb}{0, 0, 0}
\definecolor{warningcolor}{rgb}{1, 0, 1}
\definecolor{errorcolor}{rgb}{1, 0, 0}
\newenvironment{knitrout}{}{} % an empty environment to be redefined in TeX

\usepackage{alltt}
%\VignetteIndexEntry{Using SPAC}
%\VignetteEngine{knitr::knitr}

\usepackage[svgnames]{xcolor}

\usepackage[]{listings}
\lstloadlanguages{bash,R}
\lstset{
  tabsize=4,
  rulecolor=,
  language=R,
  basicstyle=\small\ttfamily,
  columns=fixed,
  showstringspaces=false,
  extendedchars=true,
  breaklines=true,
  breakatwhitespace,
  prebreak = \raisebox{0ex}[0ex][0ex]{\ensuremath{\hookleftarrow}},
  showtabs=false,
  showspaces=false,
  showstringspaces=false,
  keywordstyle=\color[rgb]{0.737,0.353,0.396},
  commentstyle=\color[rgb]{0.133,0.545,0.133},
  stringstyle=\color[rgb]{0.627,0.126,0.941},
  backgroundcolor=\color[rgb]{0.97,0.97,0.97},
}
\lstMakeShortInline{|}


\usepackage{hyperref}
\hypersetup{%
  linktocpage=false, % If true the page numbers in the toc are links
                     % instead of the section headings.
  pdfstartview=FitH,%
  breaklinks=true, pageanchor=true, %
  pdfpagemode=UseOutlines, plainpages=false, bookmarksnumbered, %
  bookmarksopen=true, bookmarksopenlevel=1, hypertexnames=true, %
  pdfhighlight=/O, %
  pdfauthor={\textcopyright\ K.~Oualkacha}, %
  colorlinks=true, %
  urlcolor=SteelBlue, linkcolor=blue, citecolor=LimeGreen, %
}

\title{A flexible copula-based approach for the analysis of secondary phenotypes in ascertained samples\\ \vspace{.5cm}
Working with SPAC}
\author{Karim Oualkacha, Genevi\`eve Lefebvre, Fod\'e Tounkara \& Celia M.T. Greenwood}
\IfFileExists{upquote.sty}{\usepackage{upquote}}{}
\begin{document}



\maketitle
\tableofcontents

\section{Introduction}
\label{sec:introduction}

This is a unified copula-based framework which tests for secondary phenotypes association and a single SNP in presence of selection bias. The method fits both retrospective and prosepctive likelihoods to handel selection bias under case-contro (CC), extreme-trait (ET) and mulitple-trait (MT) sampling designs. The dependence between primary and secondary phenotypes is modelled via copulas. Thus, SPAC is a robust method for non-normality assumptiom of the secondary trait.

SPAC is an R package that contains methods to perform genetic association of a quantitative secondary phenotype and a single SNP in presence of selection bias. SPAC is a unified copula-based framework  fits both retrospective and prosepctive likelihoods to handel selection bias under case-contro (CC), extreme-trait (ET) and mulitple-trait (MT) sampling designs. The dependence between primary and secondary phenotypes is modelled via copulas. Thus, SPAC is a robust method for non-normality assumptiom of the secondary trait; improved power is possible by appropriate modelling of the primary-secondary phenotypes joint distribution via copula models.

The main user-visible function of the package is |SPAC()| function which can be used to
analyse single-SNP/Secondary phenotype association in one go.

The main function allows for prospective and retrospective copula-based likelihoods to handle selection bias via the argument \emph{method}:
  \begin{itemize}
  \item pros (default), prospective copula-based likelihood
  \item retros, retrospective copula-based likelihood
  \end{itemize}

Of note, the retrospective method does not handle covariates while the prosepecetive-based method controls for covariates. The retrospective-based method, \emph{method="retros"}, is the computationally fastest method, and these $p$-values can be used to triage which SNPs of the genome should be re-analyzed with \emph{method="pros"}.


SPAC allows also for primary-secondary dependence modelling using five copulas

\begin{itemize}
  \item Gaussian, (default);
  \item Student, Student copula with degree of freedom equals 10;
  \item Clayton, Clayton copula;
  \item Gumbel, Gumbel copula;
  \item Frank, Frank copula.
  \end{itemize}

In order to run any of the examples below, the package needs to be installed and loaded first, of course:
\begin{knitrout}
\definecolor{shadecolor}{rgb}{0.969, 0.969, 0.969}\color{fgcolor}\begin{kframe}
\begin{alltt}
\hlkwd{library}\hlstd{(devtools)}
\hlstd{devtools}\hlopt{::}\hlkwd{install}\hlstd{()}
\end{alltt}


{\ttfamily\noindent\itshape\color{messagecolor}{\#\# Installing SPAC}}

{\ttfamily\noindent\itshape\color{messagecolor}{\#\# '/Library/Frameworks/R.framework/Resources/bin/R' --no-site-file\ \ \textbackslash{}\\\#\#\ \  --no-environ --no-save --no-restore --quiet CMD INSTALL\ \ \textbackslash{}\\\#\#\ \  '/Users/KOualkachaUQAM/Dropbox/SPAC'\ \ \textbackslash{}\\\#\#\ \  --library='/Library/Frameworks/R.framework/Versions/3.4/Resources/library'\ \ \textbackslash{}\\\#\#\ \  --install-tests}}

{\ttfamily\noindent\itshape\color{messagecolor}{\#\# }}\begin{alltt}
\hlcom{# devtools::install_github('KarimOualkacha/SPAC', build_vignettes = TRUE)}
\hlstd{devtools}\hlopt{::}\hlkwd{load_all}\hlstd{()}
\end{alltt}


{\ttfamily\noindent\itshape\color{messagecolor}{\#\# Loading SPAC}}

{\ttfamily\noindent\itshape\color{messagecolor}{\#\# Loading required package: MASS}}

{\ttfamily\noindent\itshape\color{messagecolor}{\#\# Loading required package: LaplacesDemon}}

{\ttfamily\noindent\color{warningcolor}{\#\# Warning: package 'LaplacesDemon' was built under R version 3.4.4}}

{\ttfamily\noindent\itshape\color{messagecolor}{\#\# Loading required package: VineCopula}}

{\ttfamily\noindent\color{warningcolor}{\#\# Warning: package 'VineCopula' was built under R version 3.4.4}}

{\ttfamily\noindent\itshape\color{messagecolor}{\#\# Loading required package: copula}}

{\ttfamily\noindent\color{warningcolor}{\#\# Warning: package 'copula' was built under R version 3.4.4}}

{\ttfamily\noindent\itshape\color{messagecolor}{\#\# \\\#\# Attaching package: 'copula'}}

{\ttfamily\noindent\itshape\color{messagecolor}{\#\# The following object is masked from 'package:LaplacesDemon':\\\#\# \\\#\#\ \ \ \  interval}}\begin{alltt}
\hlkwd{library}\hlstd{(SPAC)}
\end{alltt}
\end{kframe}
\end{knitrout}

\section{Loading the input data}
\label{sec:loading-input-data}

Before running an association test with one (or more) of the methods,
the following data needs to be present\footnote{The example code in
  this vignette uses files that are included in the SPAC
  package, that is why the \lstinline{package} argument to the
  \lstinline{system.file()} function is used.}:
\begin{itemize}
\item \emph{Phenotype data}; primary and secondary phenpotypes data should be present separatly in the form of
  an R vector (one value for each individual). It is up to you (as user)
  to create the vector, for example by reading it from a CSV file using R's
  |read.csv()| function or like this:
\begin{knitrout}
\definecolor{shadecolor}{rgb}{0.969, 0.969, 0.969}\color{fgcolor}\begin{kframe}
\begin{alltt}

   \hlkwd{data}\hlstd{(data,}\hlkwc{package}\hlstd{=}\hlstr{"SPAC"}\hlstd{)}
   
\hlcom{# data is .RData file contains 3 examples of data sets for CC, ET and MT designs}
\hlcom{# y1cc: primary trait under the case-control (CC) design}
  \hlkwd{head}\hlstd{(y1cc)}
\end{alltt}
\begin{verbatim}
## [1] 1 1 1 1 1 1
\end{verbatim}
\begin{alltt}
\hlcom{# y2cc: secondary trait under the case-control (CC) design}
  \hlkwd{head}\hlstd{(y2cc)}
\end{alltt}
\begin{verbatim}
## [1] 4.5444333 4.6088662 3.4706840 4.3158411 0.5313973 7.2174877
\end{verbatim}
\end{kframe}
\end{knitrout}
\item \emph{Covariate data}; this data should be present in the form
  of a matrix. Like the phenotype data it us up to you to load this
  data. For example:
\begin{knitrout}
\definecolor{shadecolor}{rgb}{0.969, 0.969, 0.969}\color{fgcolor}\begin{kframe}
\begin{alltt}
  \hlstd{data} \hlkwb{<-} \hlkwd{system.file}\hlstd{(}\hlstr{"data"}\hlstd{,}\hlstr{"data.RData"}\hlstd{,}
                            \hlkwc{package}\hlstd{=}\hlstr{"SPAC"}\hlstd{)}
  \hlkwd{load}\hlstd{(data)}
\hlcom{# data.file is .RData file contains 3 examples of data sets for CC, ET and MT designs}
\hlcom{# matrix of two confounders/covariates: one dichotmous and one continuous covariate}
  \hlkwd{dim}\hlstd{(cov.matCC)}
\end{alltt}
\begin{verbatim}
## [1] 1000    2
\end{verbatim}
\begin{alltt}
  \hlkwd{head}\hlstd{(cov.matCC)}
\end{alltt}
\begin{verbatim}
##       conf.1     conf.2
## 79222      1  2.1270291
## 48922      0  2.1488196
## 5552       0  1.9549483
## 98072      1  1.4939171
## 3987       0 -0.5533521
## 3840       1  2.5436760
\end{verbatim}
\end{kframe}
\end{knitrout}
\item \emph{Genotype data}; SNP genotype data can be in the form of
  an R vector (one value for each individual).
\begin{knitrout}
\definecolor{shadecolor}{rgb}{0.969, 0.969, 0.969}\color{fgcolor}\begin{kframe}
\begin{alltt}
  \hlstd{data} \hlkwb{<-} \hlkwd{system.file}\hlstd{(}\hlstr{"data"}\hlstd{,} \hlstr{"data.RData"}\hlstd{,}
                         \hlkwc{package}\hlstd{=}\hlstr{"SPAC"}\hlstd{)}
  \hlkwd{load}\hlstd{(data)}
\hlcom{# data is .RData file contains 3 examples of data sets for CC, ET and MT designs}
\hlcom{# SNP data for the CC design: a vector of legnth}
  \hlkwd{length}\hlstd{(markerCC)}
\end{alltt}
\begin{verbatim}
## [1] 1000
\end{verbatim}
\begin{alltt}
  \hlkwd{head}\hlstd{(markerCC)}
\end{alltt}
\begin{verbatim}
## [1] 1 1 1 0 0 1
\end{verbatim}
\end{kframe}
\end{knitrout}
\end{itemize}

\section{Analysing a single SNP}
\label{sec:analys-single-SNP}
With the phenotype data, the covariates and the genotype data loaded it is time for tests of association.

\subsection{Using SPAC for a case-control design}
\label{sec:using-SPAC-CC}
This is the simplest way to run SPAC for a single-SNP/secondary phenotype association test under the CC design:

\begin{knitrout}
\definecolor{shadecolor}{rgb}{0.969, 0.969, 0.969}\color{fgcolor}\begin{kframe}
\begin{alltt}
\hlstd{SNPresults} \hlkwb{<-} \hlkwd{SPAC}\hlstd{(}\hlkwc{y1} \hlstd{= y1cc,}
                   \hlkwc{y2} \hlstd{= y2cc,}
                   \hlkwc{G} \hlstd{= markerCC,}
                   \hlkwc{covariates} \hlstd{=} \hlkwd{as.matrix}\hlstd{(cov.matCC),}
                   \hlkwc{link} \hlstd{=} \hlstr{"probit"}\hlstd{,}
                   \hlkwc{copfit} \hlstd{=} \hlstr{"Gaussian"}\hlstd{,}
                   \hlkwc{method} \hlstd{=} \hlstr{"pros"}\hlstd{,}
                   \hlkwc{Design} \hlstd{=} \hlstr{"CC"}\hlstd{,}
                   \hlkwc{prev} \hlstd{=} \hlnum{0.1}\hlstd{)}
\end{alltt}


{\ttfamily\noindent\itshape\color{messagecolor}{\#\# Starting association analysis of the SNP...}}\begin{alltt}
\hlstd{SNPresults}
\end{alltt}
\begin{verbatim}
## $intercept.SNP.SecP
## [1] 0.63826541 0.07032754
## 
## $SNP.SecP
## [1] 0.01637158 0.05628887
## 
## $P.value.SecP
## [1] 0.7711666
## 
## $intercept.SNP.PrP
## [1] -2.24385138  0.09328779
## 
## $SNP.PrP
## [1] 0.17561800 0.06694801
## 
## $P.value.PrP
## [1] 0.008710817
## 
## $alpha
##           
## 0.4744384 
## 
## $tau
##       tau 
## 0.3146976 
## 
## $df2
##          
## 5004.195 
## 
## $AIC
## [1] 3885.296
\end{verbatim}
\end{kframe}
\end{knitrout}

\begin{itemize}
\item Here the |prev| is a scalar between 0 and 1, specifies the primary phenotype prevalance.
It is needed for the |method = "prosp"| and |Design = "CC"| or |Design = "MT"|. Default is |prev = NULL|.


\item The |link| is a character specifies the link function to be used for modelling the marginal distribution of the binary primary phenotype for the CC and MT designs. The available link functions are |link=c("probit","logit","cloglog")|. The defaut is |link = "probit"|, which th liabiltiy latent model.

\item The |copfit| is a character that selects the copula model to use for modelling primary-secondary phenptypes dependence. Can be one of the following:
\begin{itemize}
\item |copfit = "Gaussian"|, (default)
\item |copfit = "Student"|, Student copula with degree of freedom equals 10
\item |copfit = "Clayton"|, Clayton copula
\item |copfit = "Gumbel"|, Gumbel copula
\item |copfit = "Frank"|, Frank copula
\end{itemize}
\end{itemize}

\subsection{Using SPAC for an extrem-trait (ET) design}
\label{sec:using-SPAC-ET}
The next code run SPAC for a single-SNP/secondary phenotype association test under the ET design:

\begin{knitrout}
\definecolor{shadecolor}{rgb}{0.969, 0.969, 0.969}\color{fgcolor}\begin{kframe}
\begin{alltt}
\hlstd{SNPresults.ET} \hlkwb{<-} \hlkwd{SPAC}\hlstd{(}\hlkwc{y1} \hlstd{= y1et,}
                   \hlkwc{y2} \hlstd{= y2et,}
                   \hlkwc{G} \hlstd{= markeret,}
                   \hlkwc{covariates} \hlstd{=} \hlkwd{as.matrix}\hlstd{(cov.matCC),}
                   \hlkwc{copfit} \hlstd{=} \hlstr{"Gaussian"}\hlstd{,}
                   \hlkwc{method} \hlstd{=} \hlstr{"pros"}\hlstd{,}
                   \hlkwc{Design} \hlstd{=} \hlstr{"ET"}\hlstd{,}
                   \hlkwc{cutoffs} \hlstd{=} \hlkwd{c}\hlstd{(}\hlopt{-}\hlnum{1.498953}\hlstd{,}\hlnum{2.174215}\hlstd{)}
                   \hlstd{)}
\end{alltt}


{\ttfamily\noindent\itshape\color{messagecolor}{\#\# Starting association analysis of the SNP...}}\begin{alltt}
\hlstd{SNPresults.ET}
\end{alltt}
\begin{verbatim}
## $intercept.SNP.SecP
## [1] 1.02536920 0.09166647
## 
## $SNP.SecP
## [1] 0.06940319
## 
## $P.value.SecP
## [1] 0.5538218
## 
## $intercept.SNP.PrP
## [1] 0.42018677 0.05895765
## 
## $SNP.PrP
## [1] 0.08110449 0.04060297
## 
## $P.value.PrP
## [1] 0.04577073
## 
## $alpha
##           
## 0.7291998 
## 
## $tau
##      tau 
## 0.520215 
## 
## $df2
##          
## 41.31134 
## 
## $AIC
## [1] 5257.224
\end{verbatim}
\end{kframe}
\end{knitrout}

The |cutoffs| is a vector or scalar, depending on the sampling mechanism design |Design = "ET"| or  |Design = "MT"|. For the ET design the |cutoffs| is a $2\times 1$ vector, |cutoffs = c(ylb,yub)|, with
\begin{itemize}
\item ylb is the lower primary trait threshold
\item yub is the upper primary trait threshold
\end{itemize}

Of note, the ET design does not need a |link| argument since it fits a liability thresohold model for the primary phenotype.

\subsection{Using SPAC for an multiple-trait (MT) design}
\label{sec:using-SPAC-MT}
The next code run SPAC for a single-SNP/secondary phenotype association test under the MT design. Here we use |method = "Clayton"| as a copula model to model the joint distribution of the primary-secondary phenotypes:
\begin{knitrout}
\definecolor{shadecolor}{rgb}{0.969, 0.969, 0.969}\color{fgcolor}\begin{kframe}
\begin{alltt}
\hlstd{SNPresults.MT} \hlkwb{<-} \hlkwd{SPAC}\hlstd{(}\hlkwc{y1} \hlstd{= y1mt,}
                   \hlkwc{y2} \hlstd{= y2mt,}
                   \hlkwc{G} \hlstd{= markermt,}
                   \hlkwc{covariates} \hlstd{=} \hlkwd{as.matrix}\hlstd{(cov.matMT),}
                   \hlkwc{link} \hlstd{=} \hlstr{"probit"}\hlstd{,}
                   \hlkwc{copfit} \hlstd{=} \hlstr{"Clayton"}\hlstd{,}
                   \hlkwc{method} \hlstd{=} \hlstr{"pros"}\hlstd{,}
                   \hlkwc{Design} \hlstd{=} \hlstr{"MT"}\hlstd{,}
                   \hlkwc{cutoffs} \hlstd{=} \hlnum{1.865465}\hlstd{,}
                   \hlkwc{prev} \hlstd{=} \hlnum{0.1}\hlstd{)}
\end{alltt}


{\ttfamily\noindent\itshape\color{messagecolor}{\#\# Starting association analysis of the SNP...}}

{\ttfamily\noindent\color{warningcolor}{\#\# Warning in sqrt(diag(mvar)): NaNs produced}}\begin{alltt}
\hlstd{SNPresults.MT}
\end{alltt}
\begin{verbatim}
## $intercept.SNP.SecP
## [1] 0.66932760 0.07369333
## 
## $SNP.SecP
## [1] 0.05814342
## 
## $P.value.SecP
## [1] 0.6116331
## 
## $intercept.SNP.PrP
## [1] -2.2577354  0.1104643
## 
## $SNP.PrP
## [1] 0.15285477 0.06968403
## 
## $P.value.PrP
## [1] 0.02826844
## 
## $alpha
##          
## 1.225416 
## 
## $tau
##       tau 
## 0.3799249 
## 
## $df2
##          
## 50.97831 
## 
## $AIC
## [1] 3757.415
\end{verbatim}
\end{kframe}
\end{knitrout}

In the MT design, the |cutoffs| is a scalar, |cutoffs = y2ub|, with |y2ub| is the upper secondary trait threshold

\begin{thebibliography}{99}
\bibitem{Tounkara2019} Fod\'e Tounkara, Genevi\`eve Lefebvre, Celia MT Greenwood and Karim Oualkacha (2019). \emph{A flexible copula-based approach for the analysis of secondary phenotypes in ascertained samples}. Statistics in Medicine. 1-32. Under revision.
\end{thebibliography}

\end{document}


